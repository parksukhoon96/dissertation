\documentclass[11pt, a4paper]{article}
\usepackage[utf8]{inputenc}
\usepackage{fancyhdr}
\usepackage{graphicx}
\usepackage[parfill]{parskip}
\PassOptionsToPackage{hyphens}{url}\usepackage{hyperref}
\usepackage{geometry}
\usepackage[
backend=biber,
style=ieee,
]{biblatex}
\usepackage{kotex}


\pagestyle{fancy}
\fancyhf{}
\setlength{\headheight}{14pt}
\rhead{논문}
\lhead{박석훈}
\cfoot{\thepage}

\begin{document}

\title{논문}
\author{Sukhoon Park}
\maketitle

\tableofcontents

\newpage

\begin{abstract}

    
\end{abstract}


\newpage
\section{}

\subsection{}

자연주의를 자연과학과 사회과학의 논리체계가 동일하다는 명제를 의미하는 것으로 이해한다면, 꽁트와 뒤르카임에서 현대 미국사회에 이르기까지, 기능주의는 사회철학에서의 자연주의적 입장과 밀접히 관련되어 왔다. 이러한 입장에 대해 어느 누구도 꽁트가 공식화한 해석보다 더 포괄적인 것을 제공한 적이 없으며, 나는 적어도 2차 대전 이후에도 주류 사회학의 핵심 요소로 남아 있는 꽁트의 견해 중의 하나의 잔존물을 지적하고자 한다. 꽁트의 「과학의 위계」는 분석적으로나 역사적으로나 모두 적용되도록 의도된 것이었다. 즉 그것은, 생물학과 사회학간의 관계도 포함하는, 과학들간의 관계에 대해 하나의 논리적인 설명을 제공하는 것이었다. 개별과학은 위계상 아래에 있는 제 과학에 종속되며, 아울러 사실에 관한 엄밀하고 자율적인 자체의 탐구영역 (후일 뒤르카임이 열심히 반복한 개념)도 갖는다. 그러나 단편적인(horizontally) 이해가 아니라 좀더 포괄적으로(laterally) 이해를 한다면, 과학의 위계는 과학 발전의 진보――물론, 「삼단계 법칙」과 결합해서 에 대한 역사적 이해를 제공해줄 것이다. 과학은 먼저 인간의 개입과 통제로부터 가장 멀리 떨어져 있는 대상이나 사건과의 관련 속에서 발전한다. 따라서 수학과 물리학은 과학적 기반에 입각하여 가장 먼저 형성되는 분야이다. 즉 연속적인 과학의 역사는 인간사회 자체에 더욱더 근접해 오는 역사이다.


인간 행위는 과학적으로 가장 이해하기가 어렵다. 모든 인간들이기 자신의 행위를 과학적인 입장에서 보기란 아주 어렵기 때문이다. 따라서 사회학은 가장 늦게 등장한다. 여기서 이러한 일반적인 생각이 내포한 의미는 사회학은 자연과학에 비해 그 유년기적인 성격 때문에 사회학의 논리 형식에 관해 일종의 자연주의적 공식화와 결합한다는 점이다. 사회학은 인간행위의 설명에까지 실증적 정신을 확장하는 데 있어 「가장 늦게 도달」한다\footnote{앤서니 기든스. 사회이론의 주요 쟁점. 7장 현대사회이론의 전망. 319p}. \\

\footnote{The problem of solidarity theories and models. Routledge. 7. ''Modelling the interaction ritual theory of solidarity". Randall Collins and Robert hanneman} \footnote{Collins. 1975. Conflict Sociology}

고프만. 2013. 상호작용 의례: 때면 행동에 관한 에세이
뒤르켐. 1998. 직업윤리와 시민도덕. 
바우먼. 액체근대. 2009.
버거 루크만. 2014. 실재의 사회적 구성 - 지식사회학 논고
The social construction of reality a treatise in the sociology of knowledge

difference between empathy, sympathy, compassion

악셀 호네트. 인정투쟁.
뒤르켐. 종교생활의 원초적 형태들.


\subsection{}


    

\subsection{}

    
    
\subsection{}

    
\subsection{}



\end{document}